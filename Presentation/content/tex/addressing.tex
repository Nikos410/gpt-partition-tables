\begin{frame}
    \frametitle{\_}

    \begin{center}
        \vspace{1cm}
        \Huge{\textbf{Datenträger-Adressierung}}

        \vspace{0.5cm}
        \LARGE{Wie kann ich einen Bereich auf einem Datenträger eindeutig identifizieren?}
    \end{center}
\end{frame}

\begin{frame}
    \frametitle{Cylinder-Head-Sector (CHS)}

    \begin{columns}[T]
        \begin{column}{0.4\textwidth}
            
            \begin{itemize}
                \item Historisches Verfahren
                \vspace{0.2cm}
                \item Orientiert an physischem Aufbau von (HDD-) Festplatten
            \end{itemize}

            \vspace{1cm}
            \begin{alertblock}<2>{Probleme}
                \begin{itemize}
                    \item Heutzutage: Andere Festplattengeometrien / Speichertechnologien
                    
                    \vspace{0.2cm}

                    \item CHS-Adressen passen nicht (mehr) zum physischen Aufbau von Datenträgern
                \end{itemize}
            \end{alertblock}
        \end{column}

        \begin{column}{0.4\textwidth}
            \includesvg[width=\textwidth]{content/graphics/CHS.svg}
        \end{column}
    \end{columns}
\end{frame}

\begin{frame}
    \frametitle{Logical Block Adressing (LBA)}

    \begin{itemize}
        \item Datenträger wird in logische Blöcke unterteilt (Meist 512 Bytes groß)
        \item Aufsteigende Nummerierung, beginnend bei 0
        \item Festplatten-Controller "übersetzt" LBA-Nummer in tatsächlichen Speicherort
    \end{itemize}

    \vspace{1cm}
    \begin{exampleblock}<2>{Vorteile}
        \begin{itemize}
            \item Unabhänig von Art des Datenträgers
            \item Abstraktion
            \item Vereinfachte Handhabung
        \end{itemize}
    \end{exampleblock}
\end{frame}
