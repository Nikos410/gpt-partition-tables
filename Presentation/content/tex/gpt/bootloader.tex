\begin{frame}
    \frametitle{Bootloader}

    Code, der ausgeführt um ein Betriebssystem zu starten.
    
    \vspace{0.5cm}

    \pause

    \begin{columns}[T] 
        \begin{column}{0.4\textwidth}

            \textbf{MBR:} 
            Fester Bereich in der MBR Datenstruktur.

            \vspace{0.5cm}
            \begin{alertblock}{Nachteile}
                \begin{itemize}
                    \item Nur 446 Bytes verwendbar.
                    \item Nur Einstiegspunkt, verweist meist auf Code in einer Partition.
                \end{itemize}
            \end{alertblock}
        \end{column}

        \pause
        \begin{column}{0.45\textwidth}

            \textbf{GPT:} 
            Spezielle System-Partition mit definierter \texttt{partitionTypeGuid}.

            \vspace{0.5cm}
            \begin{exampleblock}{Vorteile}
                \begin{itemize}
                    \item Größe Flexibel an Bootloader-Code anpassbar.
                    \item Einträge für mehrere Betriebssysteme möglich.
                    \begin{itemize}
                        \item Multiboot-Systeme deutlich einfacher.
                    \end{itemize}
                \end{itemize}
            \end{exampleblock}
        \end{column}
    \end{columns}
\end{frame}
