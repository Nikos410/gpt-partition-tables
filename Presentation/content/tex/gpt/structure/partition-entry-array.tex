\begin{frame}
    \frametitle{Partitions-Array}

    \begin{itemize}
        \item Informationen über Partitionen.
        \item Feste Anzahl Einträge. 
        \item Einträge werden bei Anlegen der Partitionstabelle erstellt.
        \item Jeder Eintrag \textbf{kann} eine Partition beschreiben.
        \item Primäres und Backup- Partitions-Array
    \end{itemize}
\end{frame}

\begin{frame}
    \frametitle{Partitions-Array}
    \framesubtitle{Aufbau eines Eintrages}

    \begin{itemize}
        \item \textbf{partitionTypeGuid}
        \begin{itemize}
            \item GUID des Partitions-Typs.
            \item Partitions-Typ: "Verwendungszweck" der Partitionsinformationen (\textbf{nicht} verwendetes Dateisystem).
            \item Unter Linux z.B. "Daten-Partition", "RAID-Partition" oder "Swap-Partition".
            \item Indikator, ob Eintrag in Verwendung ist (eine Partition beschreibt). 
            \\Ungenutzter Eintrag: \texttt{00000000-0000-0000-0000-000000000000}
            \item GUIDs sind dezentral: Keine zentrale Liste mit allen Typen.
        \end{itemize}
    \end{itemize}
\end{frame}

\begin{frame}
    \frametitle{Partitions-Array}
    \framesubtitle{Aufbau eines Eintrages}

    \begin{itemize}
        \item \textbf{uniquePartitionGuid}
        \begin{itemize}
            \item GUID dieser Partition.
            \item Eindeutige Identifizierung der Partition.
            \item Andere Verfahren: Identifizierung meist durch Position auf dem Datenträger (z.B. "Partition 2").
        \end{itemize}

        \pause
        \item \textbf{startingLba}
        \begin{itemize}
            \item LBA-Nummer der ersten Blocks der Partition.
        \end{itemize}

        \pause
        \item \textbf{endingLba}
        \begin{itemize}
            \item LBA-Nummer der letzten Blocks der Partition.
        \end{itemize}
    \end{itemize}
\end{frame}

\begin{frame}
    \frametitle{Partitions-Array}
    \framesubtitle{Aufbau eines Eintrages}

    \begin{itemize}
        \item \textbf{attributes}
        \begin{itemize}
            \item Bits entsprechen verschiedenen Attributen.
            \item Vor allem: Wie muss das Betriebssystem / die Firmware eines Computers mit dieser Partition interagieren.
            \item Hier nicht genauer ausgeführt.
        \end{itemize}

        \pause
        \item \textbf{partitionName}
        \begin{itemize}
            \item Name der Partition.
            \item Meistens nicht verwendet.
        \end{itemize}
    \end{itemize}
\end{frame}
