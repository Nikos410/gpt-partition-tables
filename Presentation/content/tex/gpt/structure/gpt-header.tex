\begin{frame}
    \frametitle{GPT Header}

    \begin{itemize}
        \item Allgemeine Informationen über Datenträger.
        \item Primärer Header liegt \textbf{immer} in LBA 1.
        \item Backup-Header liegt am Ende des Datenträgers.
    \end{itemize}
\end{frame}

\begin{frame}
    \frametitle{GPT Header}
    \framesubtitle{Aufbau}

    \begin{itemize}
        \item \textbf{signature}
        \begin{itemize}
            \item Muss den ASCII-String \texttt{EFI PART} beinhalten.
            \item Identifizierung, ob ein GPT-Header vorliegt.
        \end{itemize}

        \pause
        \item \textbf{revision}
        \begin{itemize}
            \item Aktuell Revision 1.0: \texttt{0x00010000}
            \item Unterstützt zukünftige Erweiterbarkeit.
        \end{itemize}

        \pause
        \item \textbf{headerSize}
        \begin{itemize}
            \item Größe des Headers in Bytes.
            \item Unterstützt zukünftige Erweiterbarkeit.
        \end{itemize}
    \end{itemize}
\end{frame}

\begin{frame}
    \frametitle{GPT Header}
    \framesubtitle{Aufbau}

    \begin{itemize}
        \item \textbf{headerCrc32}
        \begin{itemize}
            \item CRC-32-Checksumme des Headers.
            \item Sicherung der Datenintegrität.
            \item Dieses Feld wird bei Berechnung auf 0 gesetzt.
        \end{itemize}

        \pause
        \item \textbf{myLba}
        \begin{itemize}
            \item LBA-Nummer des Headers.
        \end{itemize}

        \pause
        \item \textbf{alternateLba}
        \begin{itemize}
            \item LBA-Nummer des anderen Headers.
        \end{itemize}
    \end{itemize}
\end{frame}

\begin{frame}
    \frametitle{GPT Header}
    \framesubtitle{Aufbau}

    \begin{itemize}
        \item \textbf{firstUsableLba}
        \begin{itemize}
            \item LBA-Nummer des ersten für Paritionen verwendbaren Blocks.
        \end{itemize}

        \pause
        \item \textbf{lastUsableLba}
        \begin{itemize}
            \item LBA-Nummer des letzten für Paritionen verwendbaren Blocks.
        \end{itemize}

        \pause
        \item \textbf{diskGuid}
        \begin{itemize}
            \item GUID des Datenträgers.
            \item Eindeutige Identifizierung des Datenträgers.
            \item Wird bei Erstellung der Partitionstabelle neu generiert, daher richtiger: 
            \\GUID der Partitionstabelle.
        \end{itemize}
    \end{itemize}
\end{frame}

\begin{frame}
    \frametitle{GPT Header}
    \framesubtitle{Aufbau}

    \begin{itemize}
        \item \textbf{partitionEntryLba}
        \begin{itemize}
            \item Der erste Block des Partitions-Arrays.
        \end{itemize}

        \pause
        \item \textbf{numberOfPartitionEntries}
        \begin{itemize}
            \item Anzahl Einträge im Partitions-Array.
            \item \textbf{Nicht} Anzahl der Partitionen auf dem Datenträger.
        \end{itemize}

        \pause
        \item \textbf{sizeOfPartitionEntry}
        \begin{itemize}
            \item Größe eines Eintrages im Partitions-Array.
        \end{itemize}

        \pause
        \item \textbf{partitionEntryArrayCrc32}
        \begin{itemize}
            \item CRC-32-Checksumme des Partitions-Arrays.
            \item Sicherung der Datenintegrität.
        \end{itemize}
    \end{itemize}
\end{frame}
