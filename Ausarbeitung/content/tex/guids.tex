\section{GUIDs}

Wie der Name vermuten lässt, verwendet GPT \textit{GUID}s (Globally Unique Identifier), beispielsweise um Partitionen zu identifizieren.
Dabei handelt es sich um 128 Bit große Zahlen, die verwendet werden können, um Informationen eindeutig zu identifizieren.

GUIDs sind allgemeiner als \textit{UUID}s (\textbf{U}niversally \textbf{U}nique \textbf{ID}entifier) bekannt, in dieser Ausarbeitung wird allerdings die Bezeichnung GUID verwendet, um konsistent mit GPT zu sein.

\subsection{Motivation}
Aktuelle Softwaresysteme sind oft Zusammenschlüsse aus verschiedenen Anwendungen oder skalierbare Cloud-Lösungen mit mehreren Knoten / Instanzen.
In diesen Systemen ist es nicht trivial, Informationen eindeutig zu identifizieren. Einfache, aufsteigende IDs sind beispielsweise nicht praktikabel, wenn mehrere Knoten in einem solchen System IDs vergeben können, da so Überschneidungen auftreten können.

Die Synchronisierung der bereits vergebenen IDs zwischen allen Knoten ist aufwändig, fehleranfällig und beeinträchtigt die Skalierbarkeit, wodurch diese Lösung meist nicht verwendet wird.
Auch die Verwendung einer zentralen Registrierungsstelle, die als führendes System IDs vergibt, ist nicht ideal, da so ebenfalls die Skalierbarkeit beeinträchtigt wird. 
Darüberhinaus wird so ein Single Point of Failure geschaffen, der den Vorteil eines verteilten Systems wieder zunichte macht.

GUIDs lösen diese Probleme, da sie als praktisch eindeutig angesehen werden können, ohne dass eine Abstimmung mit anderen Knoten oder eine zentrale Registrierungsstelle notwendig ist.
Kollisionen sind zwar nicht ausgeschlossen, aber so unwahrscheinlich, dass sie in der Praxis nicht berücksichtigt werden müssen.


\subsection{Generierung}
Bei GUIDs wird zwischen fünf Versionen unterschieden, die jeweils unterschiedliche Methoden zur Generierung der 128 Bits verlangen.
Welche Version vorliegt, ist an den höchstwertigsten vier Bits des siebten Oktetts einer GUID zu erkennen.

Je nach Anwendungsfall können verschiedene Methoden, also GUID-Versionen, besser geeignet sein.
Eine Version wird beispielsweise auf Basis der Systemzeit generiert, während eine andere Variante komplett zufallsbasiert erstellt wird.\footnote{
    Auf die Details der Generierung wird hier nicht weiter eingegangen, das diese für die Funktionsweise von GPT nicht relevant sind und den Rahmen dieser Ausarbeitung sprengen würden.
}

\subsection{Aufbau}
Die 128 Bytes einer GUID sind in fünf Gruppen unterteilt, die in Tabelle \ref{tbl:guid_layout} aufgeführt sind.

\vspace{-0.2cm}
\begin{table}[ht]
    \begin{center}
        \begin{tabular}{|l|l|}
        \hline
        \textbf{Feld}                                                                                         & \textbf{Größe (Bytes)} \\ \hline
        \texttt{time\_low}                                                                                    & 4                        \\ \hline
        \texttt{time\_mid}                                                                                    & 2                        \\ \hline
        \texttt{time\_hi\_and\_version}                                                                       & 2                        \\ \hline
        \begin{tabular}[c]{@{}l@{}}\texttt{clock\_seq\_hi\_and\_res} \\ \texttt{clock\_seq\_low}\end{tabular} & 2                        \\ \hline
        \texttt{node}                                                                                         & 6                        \\ \hline
        \end{tabular}
    \end{center}

    \vspace{-0.6cm}

    \caption{Aufbau einer GUID der Version 1}
    \label{tbl:guid_layout}
\end{table}
\vspace{-0.5cm}

Diese Unterteilung basiert auf den Feldern der ersten, Systemzeit-basierten Version, wird aber bei allen Vesionen in der Darstellung verwendet, weshalb die Bezeichnungen der Felder nur berenzt aussagekräftig sind.\cite{uuid-rfc}

Die einzelnen Felder werden allgemein nach dem "Big-Endian"-Format gespeichert.
Bei den GUIDs, die von GPT verwendet werden, wird für die ersten 3 Felder (\texttt{time\_low}, \texttt{time\_mid} und \texttt{time\_hi\_and\_version}) allerdings stattdessen das "Little-Endian"-Format verwendet.\cite{uefi-spec}
Dies wird auch als "Mixed-Endian" bezeichnet.

Bei einer textuellen Darstellung werden die einzelnen Bytes jeweils als zweistellige Hexadezimalzahl dargestellt und die einzelnen Gruppen durch Bindestriche getrennt.
Eine GUID könnte also zum Beispiel wie folgt aussehen:

\vspace{-0.5cm}

\begin{center}
    \texttt{f81d4fae-7dec-11d0-a765-00a0c91e6bf6}
\end{center}
