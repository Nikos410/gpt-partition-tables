\section{Adressierung}
In Partionstabellen sind, unter anderem, Start und Ende der einzelnen Partitionen gespeichert.
Dafür wird eine Nomenklatur benötigt, um die Bereiche auf dem physischen Datenträger zu adressieren.

\subsection{Cylinder-Head-Sector}
Das Verfahren "Cylinder-Head-Sector" (CHS) orientiert sich am physischen Aufbau von Festplatten.
Durch Kombination von Lesekopf, Sektor und Zylinder / Spur (vgl. Abbildung \ref{fig:chs}) entsteht sozusagen ein dreidimensionales Koordinatensystem, in dem eine Position auf einem Datenträger eindeutig identifiziert werden kann.

\begin{figure}[ht]
    \centering
    \fbox{\includesvg{content/graphics/CHS}}
    \caption{Adressierung nach dem CHS-Verfahren}
    \label{fig:chs}
\end{figure}

\subsubsection{Probleme}
