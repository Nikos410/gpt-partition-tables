\section{Adressierung}
In Partionstabellen sind, unter anderem, Start und Ende der einzelnen Partitionen gespeichert.
Dafür wird eine Nomenklatur benötigt, um die Bereiche auf dem physischen Datenträger zu adressieren.

\subsection{Cylinder-Head-Sector}
Das Verfahren "Cylinder-Head-Sector" (CHS) orientiert sich am physischen Aufbau von Festplatten.
Durch Kombination von Lesekopf, Sektor und Zylinder / Spur (vgl. Abbildung \ref{fig:chs}) entsteht sozusagen ein dreidimensionales Koordinatensystem, in dem eine Position auf einem Datenträger eindeutig identifiziert werden kann.\footnote{ATA/ATAPI-5 Spezifikation - 6.2}

\begin{figure}[ht]
    \centering
    \fbox{\includesvg{content/graphics/CHS}}
    \caption{Adressierung nach dem CHS-Verfahren}
    \label{fig:chs}
\end{figure}

In modernen Festplatten werden allerdings oft komplexere Geometrien zur Anordnung der Daten verwendet, als von CHS vorgesehen sind.
Diese Geometrien werden in diesen Fällen meist von einem Festplatten-Controller, welcher den tatsächlichen Aufbau der Festplatte kennt, in CHS-Werte übersetzt, um die Daten zu adressieren.\footnote{\url{http://jdebp.eu/FGA/disc-partition-alignment.html}} 
Außerdem sind modernere Datenträger, wie zum Beispiel Flash-Speicher in SSDs, grundlegend anders aufgebaut als klassische Festplatten.

Daher entspricht die CHS-Adressierung heutzutage nur noch in den seltensten Fällen dem tatsächlichen physischen Aufbau eines Datenträgers. 
Mittlerweile wird CHS als obsolet angesehen und durch LBA\footnote{Siehe Abschnitt \ref{sec:LBA}} ersetzt.\footnote{ATA/ATAPI-6 Spezifikation - 6.2}


\subsection{Logical Block Addressing}
\label{sec:LBA}
Bei dem Verfahren "Logical Block Addressing" (LBA) werden die Daten auf einem Datenträge in 512 Byte große Blöcke (Sektoren) unterteilt, die aufsteigend nummeriert werden.\footnote{ATA/ATAPI-6 Spezifikation - 3.1.30 / 3.1.41}
Der Controller des Datenträgers, der die genaue Geometrie kennt, übersetzt die Block-Nummer dann in den physischen Speicherort der Daten.
