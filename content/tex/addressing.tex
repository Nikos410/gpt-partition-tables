\section{Datenträger-Adressierung}
\label{sec:addressing}
In Partionstabellen sind, unter anderem, Start und Ende der einzelnen Partitionen gespeichert.
Dafür wird eine Nomenklatur benötigt, um die Bereiche auf dem physischen Datenträger zu adressieren.

\subsection{Cylinder-Head-Sector}
Das Verfahren "Cylinder-Head-Sector" (CHS) orientiert sich am physischen Aufbau von Festplatten.
Durch Kombination von Lesekopf, Sektor und Zylinder / Spur (vgl. Abbildung \ref{fig:chs}) entsteht sozusagen ein dreidimensionales Koordinatensystem, in dem eine Position auf einem Datenträger eindeutig identifiziert werden kann.

\begin{figure}[ht]
    \centering
    \fbox{\includesvg{content/graphics/CHS}}
    \caption{Adressierung dines Datenträgers nach dem CHS-Verfahren}
    \label{fig:chs}
\end{figure}

\subsubsection{Probleme}
In modernen Festplatten werden oft komplexere Geometrien zur Anordnung der Daten verwendet, als von CHS vorgesehen sind.
Diese Geometrien werden in diesen Fällen meist von einem Festplatten-Controller, welcher den tatsächlichen Aufbau der Festplatte kennt, in CHS-Werte übersetzt, um die Daten zu adressieren.\cite{pollard2011}
Außerdem sind modernere Datenträger, wie zum Beispiel Flash-Speicher in SSDs, grundlegend anders aufgebaut als klassische Festplatten.

Daher entspricht die CHS-Adressierung heutzutage nur noch in den seltensten Fällen dem tatsächlichen physischen Aufbau eines Datenträgers. 
Mittlerweile wird CHS als obsolet angesehen und durch LBA ersetzt.


\subsection{Logical Block Addressing}
\label{sec:addressing:lba}
Bei dem Verfahren "Logical Block Addressing" (LBA) werden die Daten auf einem Datenträger in 512 Byte große Blöcke (Sektoren) unterteilt, die beginnend mit 0 aufsteigend nummeriert werden.
Der Controller des Datenträgers, der die genaue Geometrie kennt, übersetzt die Block-Nummer dann in den physischen Speicherort der Daten.

LBA-Block-Nummern werden häufig im Format \textit{LBA x} angegeben, wobei \textit{x} die Block-Nummer ist.
Damit würde der Anfang eines Datenträgers als LBA 0 bezeichnet werden.
