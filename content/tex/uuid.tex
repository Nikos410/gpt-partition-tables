\section{UUID}

Wie der Name vermuten lässt, verwendet GPT \textit{GUID}s (\textbf{G}lobally \textbf{U}nique \textbf{ID}entifier), beispielsweise um Partitionen eindeutig zu identifizieren.
Diese sind allgemeiner als \textit{UUID}s (\textbf{U}niversally \textbf{U}nique \textbf{ID}entifier) bekannt. 

Dabei handelt es sich um 128 Bit große Zahlen, die auch ohne zentrale Registrierungsstelle als praktisch eindeutig angesehen werden können.

\subsection{Varianten und Versionen}
\label{sec:guid:variants}

Die "Variante" einer UUID spezifiziert Informationen über dessen Aufbau.
Welche Variante vorliegt ist dem neunten Oktett (Byte) einer UUID zu entnehmen.
Abhängig von den Werten der drei höchstwertigsten Bits kann zwischen 4 Varianten unterschieden werden.

Dabei ist eine Variante für zukünftige Nutzung reserviert und zwei weitere werden für Abwärtskompatibilität zu älteren Formaten verwendet. 
Die vierte Variante wird im weiteren genauer beschrieben.

Für diese Variante wird zwischen fünf "Versionen" unterschieden.
Welche Version vorliegt, ist an den höchstwertigsten 4 Bits des siebten Oktetts zu erkennen.

Diese Versionen unterscheiden zwischen verschiedenen Methoden, durch die eine UUID generiert werden kann.
Je nach Anwendungsfall sind verschiedene Methoden, also UUID-Versionen, besser geeignet.


\subsection{Format}
\label{sec:guid:format}

Das genaue Format einer UUID ist von ihrer Version abhängig.
Bei einer Darstellung in Textform wird eine UUID in 5 Gruppen unterteilt, die in Tabelle \ref{tbl:guid_layout} angegeben sind.

\begin{table}[ht]
    \caption{Aufbau einer UUID der Version 1}
    \label{tbl:guid_layout}

    \begin{center}
        \begin{tabular}{|l|l|}
        \hline
        \textbf{Feld}                                                                                         & \textbf{Größe (Oktette)} \\ \hline
        \texttt{time\_low}                                                                                    & 4                        \\ \hline
        \texttt{time\_mid}                                                                                    & 2                        \\ \hline
        \texttt{time\_hi\_and\_version}                                                                       & 2                        \\ \hline
        \begin{tabular}[c]{@{}l@{}}\texttt{clock\_seq\_hi\_and\_res} \\ \texttt{clock\_seq\_low}\end{tabular} & 2                        \\ \hline
        \texttt{node}                                                                                         & 6                        \\ \hline
        \end{tabular}
    \end{center}
\end{table}

Diese Gruppierung basiert auf den Feldern der ersten, Timestamp-basierten Version, wird aber bei allen Vesionen in der Darstellung verwendet, weshalb die Bezeichnungen der Felder nur berenzt aussagekräftig ist.\cite{uuid-rfc}
Die einzelnen Oktette (Bytes) werden jeweils als zweistellige Hexadezimalzahl dargestellt und die einzelnen Gruppen durch Bindestriche getrennt.
Eine UUID könnte also zum Beispiel wie folgt aussehen:

\begin{center}
    \texttt{f81d4fae-7dec-11d0-a765-00a0c91e6bf6}
\end{center}
