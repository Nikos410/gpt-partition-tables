\section{Glossar}

\textbf{ASCII}:
\textit{American Standard Code for Information Exchange}. Eine weit verbreitete Zeichenkodierung.

\textbf{Auslagerungsspeicher}:
Festplattenspeicher, der als Arbeitsspeicher verwendet werden kann, wenn der eigentliche RAM belegt ist.
Im Englischen auch \textit{swap memory} genannt.

\textbf{Checksumme}:
Ein Wert, der verwendet werden kann, um die Integrität von Daten zu überprüfen.
Im Deutschen auch \textit{Prüfsumme} genannt.

\textbf{CRC}:
\textit{Cyclic Redundancy Check}.
Ein Verfahren, um aus Rohdaten eine Checksumme zu bestimmen.

\textbf{CRC32}:
Eine Variante des \textit{CRC}-Verfahrens, das 32 Bit große Checksummen generiert.

\textbf{Datenintegrität}:
Vollständigkeit, Korrektheit und Unversehrtheit von Daten. 

\textbf{Firmware}:
Sofware, die die Kommunikation zwischen Hardware und Betriebssystem / Anwendungssoftware steuert / ermöglicht.

\textbf{Multiboot-System}:
Ein Computersystem, das verschiedene Betriebssyteme starten kann.
