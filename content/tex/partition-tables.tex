\section{Partitionstabellen}
Die Speicherung von Daten auf einem Datenträger kann auf zahlreiche unterschiedliche Varianten realisiert werden.
Verschiedene Dateisysteme ordnen die gleichen Daten unterschiedlich auf einem Datenträger an.
Beispielsweise unterteilt das Dateisystem \textit{ext2} Dateien in Blöcke und verwaltet diese mit sogenannten \textit{inodes}, wohingegen bei \textit{Btrfs} die Blöcke anhand einer Baumstruktur organisiert werden. 
Jede Speicherungsmethode hat Vor- und Nachteile in verschiedenen Anwendungsfällen, weshalb es nützlich sein kann, mehrere verschiedene Dateisysteme auf einem Datenträger zu betreiben.
Dazu wird ein Datenträger in Bereiche unterteilt, die "Partitionen" genannt werden.
Aber auch aus vielen weiteren Gründen kann es hilfreich sein, einen Datenträger in verschiedene Partitionen zu unterteilen.

Partitionstabellen sind Datenstrukturen, die meist zu Beginn\footnote{
    Der "Beginn" bezeichnet hier den Bereich, den ein jeweiliger Datenträger mit LBA 0 (vgl. Abschnitt \ref{sec:addressing:lba}) adressiert.
}
eines Datenträgers vorliegen und Informationen über dessen Partitionen beinhalten.
Dies ist zwingend notwendig, damit ein Computersystem die Daten auf einem Datenträger sinnvoll lesen kann.

Außerdem beinhalten Partitionstabellen üblicherweise den sogenannten Bootloader. Dabei handelt es sich um den Code, der ausgeführt wird, wennein Computersystem von diesem Datenträger starten soll.