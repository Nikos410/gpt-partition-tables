\section{Einleitung}
Seit den Anfängen der elektronischen Datenverarbeitung ist die Speicherung von Daten ein wichtiges Thema, daher ist es kaum verwunderlich, dass die Technologie in diesem Bereich über die Jahre rasante Fortschritte erfahren hat.

Seit den 1980er Jahren gewann der \textit{Master Boot Record} (MBR) immer mehr an Bedeutung und etablierte sich als De-Facto-Standard.
Dabei handelt es sich um ein Format von Partitionstabellen.
Dieser Standard musste allerdings über die Jahre oft angepasst werden, um mit den immer neuen Anforderungen mitzuhalten.

Da dies immer häufiger zu Kompatibilitätsproblemen führte, wurde Anfang der 2000er Jahre der Standard \textit{GUID Partition Table} (GPT) entwickelt.
Dieser ist erweiterbar aufgebaut, und kann so ohne großen Aufwand an zukünftige Anforderungen angepasst werden.

Der GPT-Standard wird in dieser Ausarbeitung genauer behandelt.
Dazu werden zunächst einige grundlegende Konzepte erläutert, die für das Verständnis wichtig sind.
