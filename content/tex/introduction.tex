\section{Einleitung}
Seit den Anfängen der elektronischen Datenverarbeitung ist die Speicherung von Daten ein wichtiges Thema, daher ist es kaum verwunderlich, dass die Technologie in diesem Bereich über die Jahre rasante Fortschritte erfahren.

Eine Technologie, die seit den 1980er Jahren immer an Bedeutung gewann und sich zu einem De-Facto-Standard etabliert hat, ist der \textit{Master Boot Record} (MBR), bei dem es sich um ein Format von Partitionstabellen handelt.
Dieser Standard musste allerdings über die Jahre oft angepasst werden, um mit den immer neuen Anforderungen mitzuhalten.

Da dies immer mehr zu Kompatibilitätsproblemen führte, wurde Anfang der 2000er Jahre der Standard \textit{GUID Partition Table} (GPT) entwickelt, welcher erweiterbar aufgebaut ist, um einfach an zukünftige Anforderungen angepasst werden zu können.

Dieser Standard wird in dieser Ausarbeitung genauer behandelt.
Zunächst werden allerdings einige grundlegende Konzepte erläutert, die für das Verständnis wichtig sind.
