\section{Einleitung}
Die Speicherung von Daten auf einem Datenträger kann auf zahlreiche unterschiedliche Varianten realisiert werden.
Verschiedene Dateisysteme ordnen die gleichen Daten unterschiedlich auf einem Datenträger an.
Beispielsweise unterteilt das Dateisystem \textit{ext2} Dateien in Blöcke und verwaltet diese mit sogenannten \textit{inodes}, wohingegen bei \textit{Btrfs} die Blöcke anhand einer Baumstruktur organisiert werden. 
Jede Speicherungsmethode hat Vor- und Nachteile in verschiedenen Anwendungsfällen, weshalb es nützlich sein kann, mehrere verschiedene Dateisysteme auf einem Datenträger zu betreiben.
Dazu wird ein Datenträger in Bereiche unterteilt, die "Partitionen" genannt werden.
Aber auch aus vielen weiteren Gründen kann es hilfreich sein, einen Datenträger in verschiedene Partitionen zu unterteilen.

Damit ein Computersystem die Daten auf einem Datenträger sinnvoll lesen kann, muss klar sein, wie diese gespeichert sind.
Dafür werden Partitionstabellen verwendet, die speichern, welche Partitionen auf einem Datenträger vorliegen.

Ein Standard, der seit den 1980er-Jahren verwendet wird, ist der "Master Boot Record" (MBR).
Dieser besitzt allerdings einige technische Limitationen\footnote{Vgl. Abschnitt \ref{sec:gpt_advantages}}, weshalb eine neue Spezifikation mit dem Namen "GUID Partition Table" (GPT) entwickelt wurde.\footnote{\url{www.uefi.org/sites/default/files/resources/UEFI_Drive_Partition_Limits_Fact_Sheet.pdf}}
Diese Spezifikation wird in dieser Ausarbeitung genauer erläutert.
