\section{GUID Partition Table}
\label{sec:gpt}
In diesem Abschnitt wird der Standard \textit{GUID Partition Table} (GPT) detailliert beschrieben.
Außerdem werden an relevanten Stellen Vorteile gegenüber dem MBR-Verfahren aufgezeigt.

Im Zuge dieser Ausarbeitung wurde außerdem eine Anwendung entwickelt, um Partitions-Informationen aus einem Datenträger, der nach dem GPT-Verfahren partitioniert ist, auszulesen.
Der komplette Quellcode ist im Github-Repository \textit{Nikos410/simple-gpt-reader}\footnote{\url{https://github.com/Nikos410/simple-gpt-reader}} zu finden.
Diese Anwendung wird im folgenden als \textit{simple-gpt-reader} bezeichnet.


\subsection{Aufbau}
\label{sec:gpt:structure}

Die einzelnen Datenstrukturen, die auf einem Datenträger vorliegen, der nach dem GPT-Verfahren partitioniert ist, werden im folgenden genauer erläutert.
Die Anordnung dieser Datenstrukturen ist in Abbildung \ref{fig:gpt_layout} grafisch dargestellt.

\begin{figure}[ht]
    \includegraphics[width=\textwidth]{content/graphics/GPT_Layout.png}
    \caption{Datenstrukturen auf einem Datenträger mit GPT-Partitionierung. \cite{uefi-spec}}
    \label{fig:gpt_layout}
\end{figure}

\subsubsection{Protective MBR}
Viele ältere Betriebssyteme und Programme, die GPT nicht unterstützen, erwarten in LBA 0 einen MBR.
Wenn dieser nicht vorhanden ist, gilt der Datenträger für diese Systeme als nicht oder fehlerhaft partitioniert.
Damit der Datenträger in diesen Fällen nicht überschrieben wird, sieht GPT in LBA 0 einen MBR vor.

Dieser MBR gibt an, dass der Datenträger eine einzelne Partition beinhaltet, der den ganzen Datenträger umfasst.\footnote{
    Bei einem Datenträger, der die maximale von MBR adressierbare Größe überschreitet, beträgt die Größe dieser Partition die maximale Größe, die von MBR adressiert werden kann.
}
Dadurch muss in den zuvor erwähnten Systemen meist ein Nutzer zunächst diese Partitionen explizit löschen, bevor der Datenträger neu partitioniert werden kann.
Da so die GPT-Partitionsdaten sozusagen "geschützt" werden, wird dieser MBR als \textit{Protective MBR} bezeichnet.
Für die Funktion von GPT hat der Protective MBR ansonsten keine Bedeutung.

In Abbildung \ref{fig:protective-mbr} ist der Aufbau eines Datenträgers mit Protective MBR dargestellt.

\begin{figure}[ht]
    \includegraphics[width=\textwidth]{content/graphics/GPT_Layout_with_protective_MBR.png}
    
    \caption{Datenträger mit Protective MBR. \cite{uefi-spec}}
    \label{fig:protective-mbr}
\end{figure}


\subsubsection{GPT Header}
In LBA 1 befindet sich der sogenannte \textit{GPT Header}.
Dieser Header beinhaltet allgemeine Informationen über den Datenträger.
Im letzten Block eines Datenträgers befindet sich ein zweiter Header, um durch Redundanz Datenverlust vorzubeugen.
Der Header in LBA 1 wird dabei als \textit{primärer}, der Header am Ende als \textit{Backup}-Header bezeichnet.

In Abbildung \ref{fig:GptHeader.hpp} ist die Datenstruktur dargestellt, die im \textit{simple-gpt-reader} verwendet wird, um den Header abzubilden.
Die Bedeutung und der Zweck der einzelnen Felder werden im Folgenden genauer erläutert.

\begin{figure}[ht]
    \inputminted[baselinestretch=1.2, linenos, tabsize=4, breaklines, frame=single]{c++}{content/code/simple-gpt-reader/GptHeader.hpp}
    
    \caption{simple-gpt-reader/GptHeader.hpp (Auszug)}
    \label{fig:GptHeader.hpp}
\end{figure}

\newpage
\begin{itemize}
    \item \textbf{signature}: 
    Das erste Feld beinhaltet eine Signatur, um zu erkennen, ob es sich um einen GPT Header handelt.
    Diese Signatur muss den Wert \texttt{0x5452415020494645}\footnote{
        Dieser Wert entspricht dem ASCII-String "\texttt{EFI PART}"
    } 
    besitzen, damit es sich um einen gültigen GPT Header handelt.

    \item \textbf{revision}:
    Damit der GPT-Standard in Zukunft erweitert werden kann, ist hier die Revision gespeichert, an der man erkennen kann, wie die Daten im folgenden aufgebaut sind.
    In dieser Ausarbeitung ist die aktuellste\footnote{
        Stand Dezember 2020
    }
    Revision 1.0 beschrieben, in der dieses Feld den Wert \texttt{0x00010000} besitzt.

    \item \textbf{headerSize}:
    Die Größe des Headers in Bytes.
    Auch dies erleichtert es, den GPT-Standard in Zukunft erweitern zu können.

    \item \textbf{headerCrc32}:
    Die CRC32-Checksumme des Headers.
    Die Bytes, die für die Berechnung verwendet werden, ergeben sich aus dem Wert des Feldes \textit{headerSize}.
    Es wird diese Anzahl Bytes ab dem Anfang des Headers verwendet.
    Bei der Berechnung wird dieses Feld auf 0 gesetzt, damit der Wert dieses Feldes keinen Einfluss auf die Checksumme hat.

    Dies dient dazu, die Integrität der Daten sicherzustellen. 

    \item \textbf{reserved}:
    Dieses Feld wird nicht verwendet und muss den Wert 0 besitzen. Zukünftige Revisionen können dieses Feld verwenden.

    \item \textbf{myLba}:
    Der Block, in dem sich der aktuelle Header befindet.

    \item \textbf{alternateLba}:
    Der Block, in dem sich der andere Header befindet.
    Beim primären Header verweist dieses Feld auf den Backup-Headers, beim Backup-Header verweist es auf den primären Header.

    \item \textbf{firstUsableLba}:
    Der erste Block, der für eine Partition verwendet werden kann.
    Üblicherweise ist dies der erste Block nach dem Partitions-Array. (vgl. Abschnitt \ref{sec:gpt:structure:entry-array})

    \item \textbf{lastUsableLba}:
    Der letzte Block, der für eine Partition verwendet werden kann.

    \item \textbf{diskGuid}:
    Eine GUID, die verwendet werden kann, um den Datenträger eindeutig zu identifizieren.\cite{uefi-spec}
    Diese GUID wird beim Anlegen einer GPR-Partitionstabelle erstellt, daher kann eigentlich nur die Partitionstabelle eindeutig identifiziert werden, weshalb die Bezeichnung \textit{diskGuid} irreführend ist.

    \item \textbf{partitionEntryLba}:
    Der erste Block des Partitions-Arrays. 

    \item \textbf{numberOfPartitionEntries}:
    Die Anzahl der Einträge im Partitions-Array.
    Dies entspricht nicht der Anzahl der Partitionen, die tatsächlich auf dem Datenträger vorliegen, sondern der maximalen Anzahl der Partitionen, die im Partitions-Array gespeichert werden können.

    \item \textbf{sizeOfPartitionEntry}:
    Die Größe eines einzelnen Eintrages im Partitions-Array in Bytes.
    Zusammen mit \textit{numberOfPartitionEntries} kann so die Größe des Partitions-Arrays bestimmt werden.

    \item \textbf{partitionEntryArrayCrc32}:
    Die CRC32-Checksumme des Partitions-Arrays.
    
\end{itemize}

Der Rest des LBA-Blocks, in dem sich der Header befindet, muss den Wert 0 besitzen.

\subsubsection{Partitions-Array}
\label{sec:gpt:structure:entry-array}
Im Partitions-Array werden die Informationen zu den einzelnen Partitionen gespeichert.
Auch von dieser Datenstruktur gibt es eine primäre und ein Backup.
Der primäre Header verweist im Feld \textit{partitionEntryLba} auf das primären Partitions-Array, der Backup-Header auf das Backup-Partitions-Array.

\newpage
In Abbildung \ref{fig:GptPartitionEntry.hpp} ist die Datenstruktur dargestellt, die im \textit{simple-gpt-reader} verwendet wird, um ein Element im Partitions-Array abzubilden.
Die Bedeutung und der Zweck der einzelnen Felder werden im Folgenden genauer erläutert.

\begin{figure}[ht]
    \inputminted[baselinestretch=1.2, linenos, tabsize=4, breaklines, frame=single]{c++}{content/code/simple-gpt-reader/GptPartitionEntry.hpp}
    
    \caption{simple-gpt-reader/GptPartitionEntry.hpp (Auszug)}
    \label{fig:GptPartitionEntry.hpp}
\end{figure}

\begin{itemize}
    \item \textbf{partitionTypeGuid}:
    Eine GUID, die den Partitions-Typen identifiziert.
    Damit ist nicht das verwendete Dateisystem gemeint, sondern der Verwendungszweck der Partition.
    Beispielsweise verwendet Linux GUIDs um zwischen einfachen Daten-Partitionen, RAID-Partitionen oder Swap-Partitionen (Auslagerungsspeicher) zu unterscheiden.
    
    Da GUIDs als global eindeutig angesehen werden können, kann ein Betriebssystem-Hersteller selbstständig neue GUIDs vergeben, ohne diese bei einer zentralen Registrierungsstelle eintragen zu lassen.
    Daher gibt es auch keine vollständige Liste aller existierenden GUIDs, ein Betriebssystem kennt meist nur seine "eigenen" GUIDs.

    Anhand dieses Feldes kann auch erkannt werden, ob dieser Eintrag im Partitions-Array in Verwendung oder leer ist.
    Eine leere GUID\footnote{
        Eine "leere" GUID entspricht dem Wert 0 (\texttt{00000000-0000-0000-0000-000000000000})
    } bedeutet, dass dieser Eintrag leer ist, also keine Partition beschreibt.
    Ein Eintrag, der in Verwendung ist, also eine Partition auf dem Datenträger beschreibt, muss in diesem Feld eine nicht leere GUID besitzen.

    \item \textbf{uniquePartitionGuid}:
    Eine GUID, die diesen Partitions-Eintrag eindeutig identifiziert.
    
    Dies ist ein klarer Vorteil gegenüber dem MBR-Verfahren, bei dem Partitionen meist durch ihre Position auf dem Datenträger identifiziert werden (beispielsweise "Partition 2").
    
    Dort kann es zu Problemen kommen, wenn beispielsweise die erste Partition auf einem Datenträger in 2 Partitionen aufgeteilt wird.
    In diesem Fall erhöht sich entsprechend auch die Partitions-"Nummer" der darauf folgenden Partitionen (beispielsweise wird Partition 2 zu Partition 3).
    
    Die GUID, die von GPT verwendet wird, bleibt hingegen konstant und darf nach dem Anlegen einer Partition nicht mehr verändert werden.

    \item \textbf{startingLba}:
    Der erste Block der Partition, die von diesem Eintrag beschrieben wird.

    \item \textbf{endingLba}:
    Der letzte Block der Partition, die von diesem Eintrag beschrieben wird.

    \item \textbf{attributes}:
    Die einzelnen Bits dieses Feldes entsprechen verschiedenen Attributen.
    Diese Attribute sind für die Partitionierung selbst nicht relevant, sondern spezifizieren vor Allem, wie das Betriebssystem oder die Firmware eines Computers mit dieser Partition interagieren muss.
    Aus diesem Grund wird die Bedeutung der einzelnen Bits hier nicht weiter ausgeführt, da dies auch den Rahmen dieser Ausarbeitung sprengen würde.
    
    \item \textbf{partitionName}:
    Null-terminierter String, der einen Namen der Partition beinhaltet.
    Dieser Name muss nicht dem Namen des Dateisystems, das sich ggf. in dieser Partition befindet, entsprechen.

\end{itemize}

Wenn der Wert des Feldes \textit{sizeOfPartitionEntry} größer ist als die zuvor beschriebene Datenstruktur, muss der Rest des Partitions-Eintrags den Wert 0 besitzen und darf nicht anderweitig verwendet werden.\cite{uefi-spec}

\subsection{Bootloader}
\label{sec:gpt:bootloader}

Anders als bei anderen Verfahren besitzt GPT eine spezielle Systempartition, die den Boot-loader-Code beinhaltet.
In dieser Partition können mehrere Einträge hinzugefügt werden, was Multiboot-Systeme deutlich einfacher und vor Allem robuster macht.\cite{heise-mbr-gpt}
Bei einem Datenträger, der kein startbares Betriebssytem beinhaltet, fehlt diese Partition.

Dieser Partitions-Typ besitzt eine fest definierte GUID im Feld \textit{partitionTypeGuid}.
Firmware, die den GPT-Standard unterstützt, muss diese GUID kennen und den Bootloader-Code in dieser Partition ausführen.

Auch dieses Verfahren verbessert die Erweiterbarkeit des GPT-Standards, da die Größe dieser Partition flexibel an die Länge des Bootloader-Codes angepasst werden kann.

\newpage
\subsection{Limits}
In diesem Abschnitt werden die Grenzen beschrieben, die der GPT-Standard besitzt.

\subsubsection{Maximale Größe eines Datenträgers}
\label{sec:gpt:limits:max-partition-size}

GPT verwendet 64 Bit große Datenstrukturen, um die LBA-Block-Nummern von Anfang und Ende der einzelnen Partitionen zu speichern.
Die üblicherweise verwendete Block-Größe beträgt 512 Bytes, wodurch sich eine maximale verwendbare Größe von $ 2^{64} \cdot 512 \mathrm{B} = 8 \mathrm{ZiB} $ ergibt.

MBR hingegen verwendet 32 Bit große Datenstrukturen, wodurch sich eine maximale adressierbare Größe von $ 2^{32} \cdot 512 \mathrm{B} = 2 \mathrm{TiB} $ ergibt.

In den meisten Betriebssystemen entspricht dies auch der maximalen Größe, die ein Datenträger besitzen kann, wenn er nach dem MBR-Verfahren partitioniert ist.
Da MBR allerdings nicht die Block-Nummer des letzten Blockes einer Partition, sondern dessen Länge in Blöcken speichert, können in manchen Betriebssystemen bis zu 4TiB eines Datenträgers verwendet werden. 
Um dies zu erreichen, kann eine 2TiB große Partition angelegt werden, dessen Startblock sich an der größten möglichen LBA-Adresse befindet.
Dieses Verfahren wird allerdings nur von wenigen Betriebssystemen unterstützt, die intern Block-Nummern verwalten können, die größer als 32 Bit sind.
Diese Betriebssysteme können meist auch das GPT verwenden, weshalb dieses Verfahren in der Praxis selten angewendet wird.\cite{mbr-4tb-workaround}

Für heutige Datenträger sind oft weder 2TiB noch 4TiB ausreichend, weshalb die maximale Kapazität häufig das ausschlaggebenste Kriterium für die Verwendung von GPT ist.

\subsubsection{Maximale Anzahl Partitionen}
\label{sec:gpt:limits:max-partition-count}

GPT verwendet 32 Bits um die Anzahl der Partitionen auf einem Datenträger zu speichern (\textit{numberOfPartitionEntries}), was bedeutet dass maximal $ 2^{32} $ Partitionen verwaltet werden können.
Dieses Limit wird allerdings nicht von allen Betriebssystemen unterstützt, beispielsweise können unter Linux maximal 256 Partionen und unter Windows 128 Partitionen auf einem Datenträger verwendet werden.

MBR hingegen kann maximal vier (primäre) Partitionen verwalten.
Um dieses Limit zu erhöhen können in einer primären Partition weitere logische Partitionen angelegt werden.

\newpage
\subsection{Vorteile gegenüber MBR}
\label{sec:gpt:advantages}

In den vorherigen Abschnitten wurden viele Vorteile erläutert, die GPT gegenüber MBR besitzt.
Diese Vorteile werden an dieser Stelle noch einmal gesammelt aufgeführt.

\begin{itemize}
    \item \textbf{Maximale Größe von Partitionen:}
    GPT verwendet 64 Bit große Datenstrukturen, um LBA-Block-Nummern zu speichern, MBR 32 Bit.
    Dadurch ergeben sich nutzbare Größen von 8ZiB gegenüber 2TiB. 
    (Vgl. Abschnitt \ref{sec:gpt:limits:max-partition-size})

    \item \textbf{Maximale Anzahl von Partitionen:}
    GPT kann $ 2^{32} $, MBR nur 4 primäre Partitionen verwalten. 
    (Vgl. Abschnitt \ref{sec:gpt:limits:max-partition-count})

    \item \textbf{Erweiterbarkeit:}
    Die von GPT verwendeten Datenstrukturen sind so konzeptioniert, dass in Zukunft einfach Erweiterungen vorgenommen werden können.
    Beispielsweise werden Revision und Größe der Datenstrukturen in der Partitionstabelle gespeichert. (Vgl. Abschnitt \ref{sec:gpt:structure})

    \item \textbf{Datensicherheit:} 
    GPT verwaltet 2 redundante Partitionstabellen in verschiedenen Bereichen eines Datenträgers.
    (Vgl. Abschnitt \ref{sec:gpt:structure})
    
    \item \textbf{Datenintegrität:}
    GPT verwendet CRC32 Checksummen, um die Integrität der Daten sicherzustellen.
    (Vgl. Abschnitt \ref{sec:gpt:structure})

    \item \textbf{Bootloader:} 
    GPT besitzt eine spezielle Systempartition, die den Bootloader-Code beinhaltet.
    Dadurch wird die Verwaltung des Bootloader-Codes und die Verwendung von Multiboot-Systemen vereinfacht.
    (Vgl. Abschnitt \ref{sec:gpt:bootloader})

\end{itemize}
